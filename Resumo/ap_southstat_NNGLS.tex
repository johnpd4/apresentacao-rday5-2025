% Full instructions available at:
% https://github.com/elauksap/focus-beamertheme

\documentclass{beamer}
\usetheme{focus}

\fontfamily{cmr}\selectfont

\usepackage{ragged2e}
\usepackage[brazilian]{babel}
\usepackage{graphicx}
\usepackage[flushleft]{threeparttable}
\usepackage{tabularx}

\usepackage[utf8]{inputenc}
\usepackage{hyperref}
\usepackage{biblatex}
\addbibresource{referencias.bib}


\definecolor{main}{RGB}{128, 0, 0}
\definecolor{background}{RGB}{255, 255, 255}

\title{Comparação entre krigagem universal e NN-GLS em dados metereológicos brasileiro\\ $ $}
%\subtitle{Subtitle}
\author{\vspace{-1.5cm}Yuri Dessimon}
\institute{\vspace{-1.5cm}  Orientadora: Márcia H. Barbian \\ Departamento de Estatística \\ Instituto de Matemática e Estatística \\ Universidade Federal do Rio Grande do Sul (UFRGS)}
\date{\today}


\begin{document}
    \begin{frame}
        \maketitle
    \end{frame}

    \begin{frame}{Agenda}
	\tableofcontents
    \end{frame}
    
    % \section{Início}
    %\subsection{Contextualização}
    %\subsection{Objetivo}

    %\section{Referencial Teórico}
    %\subsection{Principais autores estudados}
    %\subsection{Conceitos e teorias relevantes}
    %\subsection{Fundamentação da escolha do tema}

    %\section{Metodologia}
    %\subsection{Dados}
    %\subsection{GNN}
        
%1
\section{Introdução}
\begin{frame}{Contextualização}
\begin{itemize}\justifying
    \vfill \item Condições meteorológicas extremas
 são fenômenos raros que ocorrem em lugares e épocas específicas, como:
    \begin{itemize}\justifying
        \vfill \item Ondas de calor e frio
        \vfill \item Secas e inundações
        \vfill \item Tempestades severas
    \vfill \end{itemize}
    \vfill \item Tais eventos são definidos por sua baixa probabilidade de ocorrência segundo a distribuição histórica.
    \vfill \item Em decorrência as mudanças climáticas a intensidade desses eventos tem aumentando.
    \vfill \end{itemize}
\end{frame}

\end{document}