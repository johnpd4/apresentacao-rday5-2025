\documentclass{2ssmeeting}

%%%%%%%%%%%%%%%%%%%%%%%%%%%%%%%%%%%%%%%%%%%%%%%%%%%%%%%%%%%%%%%%%%%%%%%%
%% POR FAVOR, NÃO FAÇA MUDANÇAS NESSE PADRÃO QUE ACARRETEM  EM
%% ALTERAÇÃO NA FORMATAÇÃO FINAL DO TEXTO
%%%%%%%%%%%%%%%%%%%%%%%%%%%%%%%%%%%%%%%%%%%%%%%%%%%%%%%%%%%%%%%%%%%%%%%%

%%%%%%%%%%%%%%%%%%%%%%%%%%%%%%%%%%%%%%%%%%%%%%%%%%%%%%%%%%%%%%%%%%%%%%%%
% POR FAVOR, ESCOLHA CONFORME O CASO
%%%%%%%%%%%%%%%%%%%%%%%%%%%%%%%%%%%%%%%%%%%%%%%%%%%%%%%%%%%%%%%%%%%%%%%%
\usepackage[brazil]{babel} % texto em português
% \usepackage[english]{babel} % texto em inglês

%\usepackage[latin1]{inputenc} % acentuação em Português ISO-8859-1
\usepackage[utf8]{inputenc} % acentuação em Português UTF-8
%%%%%%%%%%%%%%%%%%%%%%%%%%%%%%%%%%%%%%%%%%%%%%%%%%%%%%%%%%%%%%%%%%%%%%%%


%%%%%%%%%%%%%%%%%%%%%%%%%%%%%%%%%%%%%%%%%%%%%%%%%%%%%%%%%%%%%%%%%%%%%%%%
%% POR FAVOR, NÃO ALTERAR
%%%%%%%%%%%%%%%%%%%%%%%%%%%%%%%%%%%%%%%%%%%%%%%%%%%%%%%%%%%%%%%%%%%%%%%%
\usepackage[T1]{fontenc}
\usepackage{float}
\usepackage{graphics}
\usepackage{graphicx}
\usepackage{epsfig}
\usepackage{indentfirst}
\usepackage{amsmath, amsfonts, amssymb, amsthm}
\usepackage{url}
\usepackage{csquotes}
\setlength{\parindent}{0em}
% Ambientes pré-definidos

%%%%%%%%%%%%%%%%%%%%%%%%%%%%%%%%%%%%%%%%%%%%%%%%%%%%%%%%%%%%%%%%%%%%%%%%

\begin{document}

%%%%%%%%%%%%%%%%%%%%%%%%%%%%%%%%%%%%%%%%%%%%%%%%%%%%%%%%%%%%%%%%%%%%%%%%
% TÍTULO E AUTORES
%%%%%%%%%%%%%%%%%%%%%%%%%%%%%%%%%%%%%%%%%%%%%%%%%%%%%%%%%%%%%%%%%%%%%%%%
\title{Aplicação de Krigagem Ordinária em Dados Meteorológicos do Rio Grande do Sul}

% Se necessário, você deve acrescentar ou remover autores no comando abaixo. Se os autores pertencerem a mesma instituição é possível colocar apenas uma vez a instituição.

\author{
    {\large Yuri Dessimon}\thanks{yuridessimon@gmail.com}\\ 
    {\small \textit{Departamento de Estatística, Universidade Federal do Rio Grande do Sul}} \\
    {\large Pedro D. John}\thanks{johnpd4@gmail.com}\\
    {\small \textit{Departamento de Estatística, Universidade Federal do Rio Grande do Sul}}  \\
    {\large Márcia H. Barbian}\thanks{mhbarbian@gmail.com}\\
    {\small \textit{Programa de Pós-Graduação em Estatística, Universidade Federal do Rio Grande do Sul}}
}
\criartitulo
%%%%%%%%%%%%%%%%%%%%%%%%%%%%%%%%%%%%%%%%%%%%%%%%%%%%%%%%%%%%%%%%%%%%%%%%

%%%%%%%%%%%%%%%%%%%%%%%%%%%%%%%%%%%%%%%%%%%%%%%%%%%%%%%%%%%%%%%%%%%%%%%%
% RESUMO
%%%%%%%%%%%%%%%%%%%%%%%%%%%%%%%%%%%%%%%%%%%%%%%%%%%%%%%%%%%%%%%%%%%%%%%%

\vspace{1cm}

\textbf{RESUMO}

\vspace{0.1cm}

A estimativa da precipitação no Rio Grande do Sul, após o desastre climático de 2024, é essencial para identificar desvios em relação aos padrões históricos e subsidiar modelos de risco hidrometeorológico. Tal quantificação fornece suporte técnico-científico à formulação de políticas públicas voltadas à mitigação de impactos e à adaptação às mudanças climáticas. O objetivo desse trabalho é estimar a distribuição espacial da precipitação e temperatura no estado do Rio Grande do Sul no mês de fevereiro de 2025, para isso serão considerados dados meteorológicos coletados em 40 centros de monitoramento espalhados no  estado. Para a estimativa da precipitação e da temperatura, será utilizado o modelo geoestatístico clássico, tal que a variável de interesse \( Z(s) \) na localização \( s \in D \subset \mathbb{R}^2 \) é considerada como a realização de um processo estocástico, com média constante desconhecida \( \mu \) e estrutura de covariância dependente apenas da distância entre as observações. Sob essas condições, utiliza-se a krigagem ordinária, um estimador linear e não viciado. O trabalho foi realizado utilizando o software R, o modelo geoestatístico foi estimado através do pacote  gstat, para confecção dos mapas foi utilizado o pacote leaflet. Como resultado observou-se um padrão espacial com maiores valores de precipitação concentrados na porção leste do estado, enquanto as temperaturas mais elevadas são obervadas nas regiões oeste e noroeste. Essa configuração está em consonância com padrões climáticos regionais, nos quais a influência oceânica contribui para maior umidade no leste, enquanto o interior do estado apresenta maior continentalidade e, consequentemente, temperaturas mais elevadas. A estimação espacial dessas variáveis é de fundamental importância para o monitoramento climático, contribuindo para a identificação de áreas de risco, a formulação de estratégias de mitigação de impactos extremos e o planejamento de ações adaptativas frente às mudanças climáticas. Agradecemos à Fundação de Amparo à Pesquisa do Estado do Rio Grande do Sul (FAPERGS) pelo apoio financeiro, por meio do Projeto nº 24/2551-0002361-5, contemplado no Edital 06/2024 – Programa de Pesquisa e Desenvolvimento Voltado a Desastres Climáticos. Este suporte foi fundamental para a realização e o avanço desta pesquisa.

\vspace{1.5em}
\vspace{1.5em}
\vspace{1.5em}


{\bf Palavras-chave:} Krigagem; Geoestatística; Semivariograma.

\end{document}




